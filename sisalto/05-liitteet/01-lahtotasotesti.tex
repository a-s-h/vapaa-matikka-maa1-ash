\chapter{Lähtötasotesti}
%vastaukset voisi olla jossain takana eikä tässä ettei kovin moni lukisi niitä ennen testiä

%\todo[inline]{On mietittävä, mikä on lähtötasotestin tarkoitus. Tarvitaanko sitä tässä oppikirjassa vai ei. Luultavasti voisi poistaa. Onko sen teettäminen järkevää ajankäyttöä? Mikä on sen rooli? Kannustava alkuverryttely, opettajalle koko vuosiluokan tarkasteluun työkalu vai jotain muuta?} no ei se negatiivista voi olla, että tämä on. Käyttäkööt, jos käyttävät. Hommas selvä. –Joonas%

\section*{Lähtötasotesti kurssille MAA1}

\todo{Muokataan tehtäväympäristöä niin, etteivät nämä jatka vanhaa tehtävänumerointia.}
\begin{tehtava}
Laske. Merkitse välivaiheet näkyviin. 
\begin{enumerate}
\item $2+3\cdot 2-1\cdot5$
\item $(2+1)^2+\frac{4-2}{2}$
\end{enumerate}
\begin{vastaus}
\item $2+6-5=3$
\item $3^2+\frac{2}{2}=9+1=10$
\end{vastaus}
\end{tehtava}

\begin{tehtava}
Laske. Merkitse välivaiheet näkyviin. 
\begin{enumerate}
\item $4\cdot \frac{2}{5} + \frac{2}{3}\cdot \frac{3}{5}$
\item $\frac{2}{5} : \frac{3}{2}$
\end{enumerate}
\begin{vastaus}
\item $\frac{8}{5} + \frac{2}{5}=\frac{10}{5} = 2$
\item $\frac{2}{5} \cdot \frac{2}{3}=\frac{4}{15}=$
\end{vastaus}
\end{tehtava}

\begin{tehtava}
\begin{enumerate}
\item Kuinka paljon on 5~\% luvusta 40?
\item Kuinka monta prosenttia 2 on luvusta 4?
\end{enumerate}
\begin{vastaus}
\item 2
\item 50~\%
\end{vastaus}
\end{tehtava}

\begin{tehtava}
Sievennä.
\begin{enumerate}
\item $x + 2x+x^2$
\item $2(4x+1)$
\end{enumerate}
\begin{vastaus}
\item $2x +x^2$
\item $8x+2$
\end{vastaus}
\end{tehtava}

\begin{tehtava}
\begin{enumerate}
\item Matka ja aika ovat suoraan verrannollisia. Jos matka kaksinkertaistuu, niin miten käy ajalle?
\item Sievennä $-3x+4x^2-2x+x^2$.
\end{enumerate}
\begin{vastaus}
\item Aika kaksinkertaistuu
\item $5x^2-5x$
\end{vastaus}
\end{tehtava}

\begin{tehtava}
Ratkaise yhtälöt.
\begin{enumerate}
\item $2x+5 = -1$
\item $x^2 = 9$
\end{enumerate}
\begin{vastaus}
\item $x=-3$
\item $x=-3$ tai $x=3$
\end{vastaus}
\end{tehtava}

\begin{tehtava}
Funktiot on määritelty seuraavasti: $f(x)= x^2+3x$ ja $g(x)=2x-8$.
\begin{enumerate}
\item Laske $f(-2)$.
\item Millä $x$:n arvolla $g(x)=0$?
\end{enumerate}
\begin{vastaus}
\item $(-2)^2+3\cdot(-2)=-2$
\item $x=4$
\end{vastaus}
\end{tehtava}


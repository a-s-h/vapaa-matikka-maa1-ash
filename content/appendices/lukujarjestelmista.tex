\section{Lukujärjestelmistä}

\laatikko{Tämä teksti kaipaa huomiota, kirjoita selkeämmäksi kiitos!}

Kymmenjärjestelmää kutsutaan \emph{desimaalijärjestelmäksi}, koska siinä hyödynnetään kymmentä eri numeromerkkiä.
Mikään ei kuitenkaan pakota käyttämään juuri kymmentä merkkiä. 
Muita yleisesti käytössä olevia järjestelmiä ovat 2-järjestelmä eli binäärijärjestelmä ja 16-järjestelmä eli heksadesimaalijärjestelmä, joita käytetään digitaalisen informaation tallentamiseen ja käsittelemiseen. Binäärijärjestelmässä luvun muodostavia numeromerkkejä kutsutaan biteiksi. Bitti voi olla joko päällä (1) tai pois päältä (0), ja toteutus tietokoneessa vastaa esimerkiksi sitä, että johtimessa kulkee virta (1) tai ei (0). Useampaa järjestelmää käytettäessä merkitään kantaluku luvun jälkeen alaindeksinä. Esimerkiksi luku yhdeksäntoista voidaan merkitä $19_{10}$, $10011_{2}$ tai $13_{16}$ käyttäen desimaali-, binääri- tai heksadesimaalijärjestelmää.

\begin{esimerkki}
\begin{align*}
19_{10} &= 1 \cdot 10 + 9 \\
10011_{2} &= 1 \cdot (2 \cdot 2 \cdot 2 \cdot 2) + 0 \cdot (2 \cdot 2 \cdot 2) + 0 \cdot (2 \cdot 2) + 1 \cdot 2 + 1 \\
13_{16} &= 1 \cdot 16 + 3
\end{align*}
\end{esimerkki}

Kuusitoistajärjestelmässä tarvitaan vielä kuusi uutta numeromerkkiä. Tavaksi on vakiintunut käyttää kirjainmerkkejä $\mathrm{A, B, C, D, E}$ ja $\mathrm{F}$. Ne vastaavat lukuja $10, 11, 12, 13, 14$ ja $15$. Yleisesti $n$-järjestelmässä käytetään $n$:ää kappaletta eri merkkejä, jotka merkitsevät lukuja nollasta lukuun $n-1$.

\begin{esimerkki}
$F4C_{16} = F \cdot (16 \cdot 16) + 4 \cdot 16 + C = 15 \cdot (16 \cdot 16) + 4 \cdot 16 + 12 = 3916_{10}$
\end{esimerkki}
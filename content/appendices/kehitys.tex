\section{Kirjan kehitykseen osallistuminen}

Tässä liitteessä kuvataan, kuinka voit osallistua kirjan kehittämiseen.

Työkalut ja niiden käyttötarkoitus lyhyesti:

\begin{enumerate}

\item Virheraportit https://github.com/Oppikirjamaraton/oppikirjamaraton-maa1/issues
\item Versionhallinta https://github.com/Oppikirjamaraton/oppikirjamaraton-maa1
\item Git-työkalu omalle koneelle: http://git-scm.com/
\item Koodin muokkaamiseen http://www.xm1math.net/texmaker/
\item Tekstin prosessointiin \LaTeX (eri distribuutioita)
\item Kuvituksiin mm. GeoGebra: http://www.geogebra.org/cms/

\end{enumerate}

\subsection*{Tee tunnus github.com:iin}

Tee tunnus githubiin. Github.com on palvelu, joka antaa ilmaista palvelintilaa avoimen lähdekoodin projekteille.

\laatikko{
https://github.com/signup/free
}

\subsection*{Forkkaa projekti ja aloita työskentely}

Forkkaaminen (fork) on github-toimenpide, joka tarkoittaa sitä että teet oman versiosi kirjasta. Työstäessäsi muutoksia, teet ensin muutokset omaan versioosi jonka jälkeen lähetät muutostoiveen (pull request) pääprojektin hallinnoijalle.

\begin{enumerate}

\item Kirjaudu sisään github.com:iin ja mene osoitteeseen https://github.com/Oppikirjamaraton/oppikirjamaraton-maa1/
\item Etsi fork -nappula ja paina sitä (oikea ylänurkka)
\item Onnittelut! Olet luonut oman kopioisi projektista. 

\end{enumerate}

Voit muokata omaa versiotasi suoraan githubin nettisivuilta (Edit). Kun olet tehnyt muutoksen nettikäyttöliittymässä, kuvaile yhdellä rivillä mitä teit ja paina Commit changes.

\subsection*{Pull request}

Kun olet tehnyt muutoksen omaan kopioosi, voit lähettää muutosehdotukset suoraan pääkehittäjälle tekemällä Pull Request-muutosehdotuksen. Tämä tapahtuu painamalla Pull Request nappulaa githubissa oman kopiosi sivulla.

Voit tarkastella tekemiäsi muutoksia githubista käsin. Kirjoita pull requestillesi hyvä ja ytimekäs kuvaus.

\laatikko{
Pull requestit syvällisemmin: https://help.github.com/articles/using-pull-requests
}

Pienet muutokset, kuten kirjoitusvirheet, voit tehdä suoraan githubin nettikäyttöliittymässä. Isompia muutoksia varten kannattaa asentaa \LaTeX ja git omalle koneellesi, niin voit kokeilla miltä lopputulos näyttää ja muokata koodia tehokkaammin.

\subsection*{Versionhallinta omalle koneelle: git}

Jotta pääsisit paremmin muokkaamaan ja tarkastelemaan koodia, kannattaa asentaa omalle koneelle versionhallintatyökalu. Github tarjoaa omat työkalunsa eri käyttöjärjestelmille, mutta voit myös asentaa alkuperäisen git-versiohallintajärjestelmän osoitteesta http://git-scm.com/

Tässä oppaassa käydään läpi komentorivityökalu-komennot. Vastaavat komennot graafisista työkaluista on helppo löytää. Kaikki komennot suoritetaan pääkansiosta (oppikirjamaraton-maa1). Korvaa tagit (<tagi>) omilla tiedoillasi.

\subsubsection{Tiedostot omalle koneelle}

\begin{verbatim}
git clone https://github.com/<oma tunnus>/oppikirjamaraton-maa1.git
cd oppikirjamaraton-maa1
\end{verbatim}

\subsubsection{Päähaaran asettaminen}

Tämä toiminto pitää tehdä, jotta voit tehdä yhteistyötä projektin vetäjän kanssa.

\begin{verbatim}
git remote add upstream https://github.com/Oppikirjamaraton/oppikirjamaraton-maa1.git
\end{verbatim}

\subsubsection{Muutoksien päivittäminen päähaarasta}

\begin{verbatim}
git fetch upstream
git merge upstream/master
\end{verbatim}

\subsubsection{Omien muutoksien lähettäminen}

\begin{verbatim}
git commit -m "<muutoksen kuvaus>" -a
git push origin master
\end{verbatim}

Kun olet saanut valmiiksi muutokset, tee pull request github.com:n kautta.

\subsubsection{Lähdekoodin katsominen ja muokkaaminen}

Voit käyttää koodin muokkaamiseen mitä tahansa tekstieditoria, vaikkapa notepadia. Erillisen tähän tarkoituksene soveltuvan ohjelman asentaminen on kuitenkin suositeltavaa, tässä eräs hyväksi koettu ohjelma: http://www.xm1math.net/texmaker/

\subsection*{lopullisen version kääntämiseen \LaTeX}

\LaTeX on työkalu, jolla saat muutettua tekstimuotoisen lähdekoodin kauniiseen muotoon esim. PDF-formaattiin.

\begin{itemize}

\item Mac: http://www.tug.org/mactex/
\item Windows: http://miktex.org/
\item Ubuntu: aptitude install latex

\end{itemize}

Komentoriviltä kirja käännetään seuraavilla komennoilla:

\begin{verbatim}
cd sisalto
pdflatex kokokirja.tex
\end{verbatim}

Lyhyt \LaTeX -muistilehtiö: http://www.stdout.org/~winston/latex/latexsheet-a4.pdf


\section{Matematiikka ja kognitio}

Ihmisillä ja eläimillä on luonnostaan matemaattisia taitoja. Monet niistä, esimerkiksi lukumäärien laskeminen, ovat monimutkaisia ajatusprosesseja, jotka kehittyvät lapsuudessa – toisilla aiemmin, toisilla myöhemmin. Koulussa opeteltava peruslaskento ja myös matematiikka tieteenalana rakentuvat tämän biologisen osaamisen päälle. Laskeminen itsessään on vain eräs matematiikan osa-alue, eikä kaikki matematiikka ole laskemista.

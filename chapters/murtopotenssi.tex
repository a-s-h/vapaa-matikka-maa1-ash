\section{Murtopotenssi}

Seuraavaksi tutkitaan potenssin käsitteen laajentamista tilanteeseen, jossa eksponenttina on murtoluku.
Esimerkiksi voidaan pohtia, mitä tarkoittaa merkintä $2^\frac{1}{3}$? Potenssin laskusääntöjen perusteella
\[
(2^{m})^n = 2^{mn},\textrm{ kun }m,n\in \Z.
\]
Siten on luonnollista ajatella, että murtopotenssille $2^\frac{1}{3}$ pätee
\[
(2^\frac{1}{3})^3 = 2^\frac{3}{3} = 2^1=2.
\]
Koska luvun $2$ kuutiojuuri toteuttaa yhtälön $(\sqrt[3]{2})^3=2$, täytyy siis asettaa $2^\frac{1}{3}=\sqrt[3]{2}$. Yleisemmin asetetaan $a^\frac{1}{n} =\sqrt[n]{a}$, kun $n$ on positiivinen kokonaisluku ja $a\ge 0$.

Kaavan on edelleen luontevaa ajatella yleistyvän niin, että esimerkiksi
\[
(2^{\frac{1}{3}})^2 = 2^{\frac{2}{3}}.
\]
Yleisemmin otetaan murtopotenssin $a^\frac{m}{n}$ määritelmäksi
\[
a^\frac{m}{n} = (a^{\frac{1}{n}})^m = (\sqrt[n]{a})^m,
\]
kun $m$ ja $n$ ovat kokonaislukuja ja $n>0$. 

\laatikko{{\bf Murtopotenssimerkinnät}
\[
a^\frac{1}{n} = \sqrt[n]{a},\textrm{ kun }a\geq 0. 
\]
Erityisesti $a^\frac{1}{2}=\sqrt{a}$.
\[
a^\frac{m}{n} =  (\sqrt[n]{a})^m,\textrm{ kun }a > 0. % Issue 68 fiksattu.
\]
}

%\laatikko{Murtopotenssimerkintä: }

Kun murtolukueksponentit määritellään näin, kaikki aikaisemmat potenssien
laskusäännöt ovat sellaisenaan voimassa myös niille. Esimerkiksi kaavat
\[ a^q\cdot a^q = a^{p+q}, \quad (a^p)^q = a^{pq}, \quad (ab)^q=a^qb^q \]
pätevät kaikille rationaaliluvuille $p$ ja $q$.  Näiden kaavojen todistukset on esitetty liitteessä \ref{pot_todistukset}.

{\bf Huomautus määrittelyjoukosta}. Murtopotenssimerkintää käytettäessä vaaditaan, että $a\geq 0$ myös silloin, kun $n$ on pariton. Syy tähän on seuraava. Esimerkiksi $\sqrt[3]{-1}=-1$, koska $(-1)^3=-1$, mutta lauseketta $(-1)^\frac{1}{3}$ ei ole tällöin määritelty. Murtopotenssimerkinnän määrittelystä voi tällöin seurata yllättäviä ongelmia:
\[
 -1 = \sqrt[3]{-1} = (-1)^\frac{1}{3} = (-1)^\frac{2}{6}
= ((-1)^2)^\frac{1}{6} = 1^\frac{1}{6} = \sqrt[6]{1} = 1. 
\]
Luvun $\frac{1}{3}$ lavennus muotoon $\frac{2}{6}$ on ongelman ydin, mutta murtolukujen lavennussäännöistä luopuminen olisi myös hankalaa. Siksi sovitaan, ettei murtopotenssimerkintää käytetä, jos kantaluku on negatiivinen.

Nolla kantalukuna taas aiheuttaa ongelmia, jos eksponentti on negatiivinen. Esimerkiksi
\[
 0^{-\tfrac{1}{2}}=\frac{1}{0^\frac{1}{2}}=\frac{1}{0}, % Issue 68 fiksattu.
\]
jota ei ole määritelty.

\begin{esimerkki}
Muuta lausekkeet $\sqrt[5]{3}$ ja $(\sqrt[4]{a})^7$ murtopotenssimuotoon.

{\bf Ratkaisu.}

$\sqrt[5]{3} = 3^\frac{1}{5}$, \\
$(\sqrt[4]{a})^7 = (a^\frac{1}{4})^7=a^\frac{7}{4}$
\end{esimerkki}

\begin{esimerkki}
Sievennä lauseke $8^\frac{2}{3}$.

{\bf Ratkaisu.}
 $8^\frac{2}{3} = (\sqrt[3]{8})^2 = 2^2 = 4.$
\end{esimerkki}

\begin{tehtavasivu}

Muuta lausekkeet murtopotenssimuotoon.

\begin{tehtava}
	\begin{alakohdat}
		\alakohta{$\sqrt[3]{a}$} 
		\alakohta{$\sqrt[6]{a}$} 
	\alakohta{$\sqrt[n]{a}$}
	\end{alakohdat}
	\begin{vastaus}
		\begin{alakohdat}
			\alakohta{$a^\frac{1}{3}$} 
			\alakohta{$a^\frac{1}{6}$}
			\alakohta{$a^\frac{1}{n}$}
		\end{alakohdat}
	\end{vastaus}
\end{tehtava}

\begin{tehtava}
	\begin{alakohdat}
		\alakohta{$(\sqrt[3]{b})^6$}
		\alakohta{$(\sqrt[6]{b^3})$}
		\alakohta{$(\sqrt[5]{b})^2$}
		\alakohta{$(\sqrt[16]{\ddot{o}^4})$}
	\end{alakohdat}
	\begin{vastaus}
		\begin{alakohdat}
			\alakohta{$b^2$}
			\alakohta{$b^\frac{1}{2}$}
			\alakohta{$b^\frac{2}{5}$}
			\alakohta{$\ddot{o}^\frac{1}{4}$}
		\end{alakohdat}
	\end{vastaus}
\end{tehtava}

Sievennä.
\begin{tehtava}
	\begin{alakohdat}
		\alakohta{$x^\frac{1}{5}$}
		\alakohta{$x^\frac{4}{3}$}
		\alakohta{$x^\frac{8}{4}$}
		\alakohta{$x^\frac{25}{100}$}
\end{alakohdat}
	\begin{vastaus}
		\begin{alakohdat}	
			\alakohta{$(\sqrt[5]{x})$}
			\alakohta{$(\sqrt[3]{x})^4$}
			\alakohta{$x^2$}
			\alakohta{$(\sqrt[4]{x})$}
		\end{alakohdat} 
	\end{vastaus}
\end{tehtava}

\begin{tehtava}
	\begin{alakohdat}
		\alakohta{$9^\frac{1}{2}$}
		\alakohta{$8^\frac{1}{3}$}
		\alakohta{$4^\frac{3}{2}$}
		\alakohta{$81^\frac{3}{4}$}
	\end{alakohdat}
	\begin{vastaus}
		\begin{alakohdat}
			\alakohta{$3$}
			\alakohta{$2$}
			\alakohta{$8$}
			\alakohta{$27$} 
		\end{alakohdat}
	\end{vastaus}
\end{tehtava}

\begin{tehtava}
	\begin{alakohdat}
		\alakohta{$x^{-\frac{1}{3}}$} 
		\alakohta{$x^\frac{5}{-2}$}
		\alakohta{$x^{3 \frac{1}{2}}$}
		\alakohta{$x^{2 \frac{-4}{-7}}$}
	\end{alakohdat}
	\begin{vastaus}
		\begin{alakohdat}
			\alakohta{$\frac{1}{\sqrt[3]{x}}$} 
			\alakohta{$\frac{1}{\sqrt{x^5}}$} 
			\alakohta{$\sqrt{x^3}$} 
			\alakohta{$\sqrt[7]{x^8}$} 
		\end{alakohdat}
	\end{vastaus}
\end{tehtava}

\begin{tehtava}
	\begin{alakohdat}
		\alakohta{$4^\frac{3}{4}$}
		\alakohta{$2^\frac{5}{1}$} 
		\alakohta{$16^\frac{2}{3}$}
		\alakohta{$5^\frac{5}{3}$}
	\end{alakohdat}
	\begin{vastaus}
		\begin{alakohdat}
			\alakohta{$2\sqrt{2}$}
			\alakohta{$32$} 
			\alakohta{$(\sqrt[3]{16})^2$} 
			\alakohta{$5(\sqrt[3]{5})^2$} 
		\end{alakohdat}
	\end{vastaus}
\end{tehtava}

\begin{tehtava}
	\begin{alakohdat}
	$\boldsymbol{[\star]}$ Oletetaan, että $a\leq0$ \\
		\alakohta{$\frac{\sqrt[3]{a^2}-\sqrt[3]{a}}{\sqrt[3]{a}}$}
		\alakohta{$\sqrt{\frac{\sqrt{a}^2\sqrt[3]{a^6}}{4a}}$}
		\alakohta{$\sqrt[7]{\frac{a-\sqrt{a}^2}{a^0}}$} 
		\alakohta{$\sqrt{\sqrt{a^3}\left(\frac{a^7}{\sqrt[5]{a^3}}\right)^0\sqrt{\frac{a^5}{a^2}}}$}
	\end{alakohdat}
	\begin{vastaus}
		\begin{alakohdat}
			\alakohta{$\sqrt[3]{a}-1$} 
			\alakohta{$\frac{1}{2}a$} 
			\alakohta{$1$} 
			\alakohta{$a\sqrt{a}$}
		\end{alakohdat}
	\end{vastaus}
\end{tehtava}

Muuta murtopotenssimuotoon.

\begin{tehtava}
	\begin{alakohdat}
		\alakohta{$\sqrt{\sqrt{k}}$} 
		\alakohta{$\sqrt[3]{\sqrt[4]{k}}$}
		\alakohta{$\sqrt{m\sqrt{m}}$}
		\alakohta{$\sqrt[5]{m\sqrt[7]{m}}$}
	\end{alakohdat}
	\begin{vastaus}
		\begin{alakohdat}
			\alakohta{$k^\frac{1}{4}$} 
			\alakohta{$k^\frac{1}{12}$} 
			\alakohta{$m^\frac{3}{4}$} 
			\alakohta{$m^\frac{8}{35}$}
		\end{alakohdat}
	\end{vastaus}
\end{tehtava}

\begin{tehtava}
	\begin{alakohdat}
		\alakohta{$\sqrt[3]{\sqrt[3]{\alpha}^2}$} 
		\alakohta{$\sqrt[5]{q^2\sqrt{q}}$} 
		\alakohta{$\left(\sqrt{å^4\sqrt{å}}\right)^3$}
	\end{alakohdat}
	\begin{vastaus}
		\begin{alakohdat}
			\alakohta{$\alpha^\frac{2}{9}$}
			\alakohta{$q^\frac{1}{2}$} 
			\alakohta{$å^\frac{27}{4}$}
		\end{alakohdat}
	\end{vastaus}
\end{tehtava}
 
\begin{tehtava}
	\begin{alakohdat}
	$\boldsymbol{[\star]}$ Tutki laskimella, miten laskutoimitukset
	\[ x^x, \quad x^{x^x}, \quad x^{x^{x^x}}, \quad \ldots\]
	käyttäytyvät, kun eksponenttien määrää kasvatetaan mielivaltaisen
	suureksi.\\
		\alakohta{Tutki arvoja $x=1,3$, $x=1,7$ ja $x=0,05$.}
		\alakohta{Millä luvun $x$ positiivisilla arvoilla potenssitornien arvot}
			vakiintuvat tiettyyn lukuun?
	\end{alakohdat}
	\begin{vastaus}
		\begin{alakohdat}
			\alakohta{Kun $x = 1,3$, tornien arvot vakiintuvat lukuun $1,4709\ldots$;
				kun $x = 1,7$, tornien arvot kasvavat mielivaltaisen suuriksi;
				kun $x=0,05$, tornien arvot vuorottelevat lukujen $0,136\ldots$
				ja $0,664\ldots$ välillä.}
			\alakohta{Arvot vakiintuvat, kun $0,00659 \ldots \leq x \leq 1,444667\ldots$;
				tämän voi esittää muodossa $\frac{1}{e^e} \leq x \leq \sqrt[e]{e}$, missä
				$e$ on luku nimeltään Neperin luku. Desimaalilukuna $e = 2,718281828459 \ldots$ .}
		\end{alakohdat}
\end{vastaus}
\end{tehtava}
\begin{tehtava}
	$\boldsymbol{[\star]}$ Potenssitornissa $x^{x^{x^{x^{\mathstrut^{.^{.^{.}}}}}}} $
		on äärettömän monta päällekkäistä eksponenttia. Yhtälöllä
	$ x^{x^{x^{x^{\mathstrut^{.^{.^{.}}}}}}} =2$
		on yksi ratkaisu. Etsi se.
\begin{vastaus}
	$x = \sqrt{2}$
\end{vastaus}
\end{tehtava}
\end{tehtavasivu}

\end{tehtavasivu}

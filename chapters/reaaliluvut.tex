\section{Reaaliluvut}

Kaikki käyttämämme luvut eivät ole rationaalilukuja. Esimerkiksi $\sqrt{2}$ ei ole esitettävissä murtolukuja. Se tulee kuitenkin vastaan esimerksi suorakulmaisessa kolmiossa, jonka kateetit ovat pituudeltaan 1.

\laatikko{ TÄHÄN KUVA suorakulmaisesta kolmiosta, jonka sivut ovat
$1$, $1$ ja $\sqrt{2}$}

Luku $\sqrt{2}$ on \termi{irrationaaliluku}{irrationaaliluku}. Toinen tuttu peruskoulusta tuttu irrationaaliluku on ympyrän kehän pituuden suhde halkaisijaan: $\pi$. Irrationaaliluvut voidaan sijoittaa
lukusuoralle rationaalilukujen tapaan.

%%%%TÄMÄ ON KUVA LUKUSUORALLA OLEVASTA PIISTÄ JA SQRT 2:STA%%%
\begin{scriptsize}

\end{scriptsize}
\definecolor{ffqqqq}{rgb}{1,0,0}
\begin{tikzpicture}[line cap=round,line join=round,>=triangle 45,x=6.0cm,y=6.0cm]
\pgfkeys{/pgf/number format/.cd,fixed,precision=2,use comma} % Pisteen tilalle pilkku, ohje
% http://tex.stackexchange.com/questions/31276/number-format-in-pgfplots-axis ja sieltä
% viittaus pgf:n manuskaan ftp://ftp.funet.fi/pub/TeX/CTAN/graphics/pgf/base/doc/generic/pgf/pgfmanual.pdf sivu 547
\draw[->,color=black] (1.2,0) -- (3.3,0);
\foreach \x in {1.2,1.3,1.4,1.5,1.6,1.7,1.8,1.9,2,2.1,2.2,2.3,2.4,2.5,2.6,2.7,2.8,2.9,3,3.1,3.2,3.3}
\draw[shift={(\x,0)},color=black] (0pt,2pt) -- (0pt,-2pt) node[below] {\footnotesize \pgfmathprintnumber{\x}};
\clip(1.2,-0.1) rectangle (3.3,0.3);
\draw (0.5,2.1) node[anchor=north west] {$\sqrt[]{2}$};
\draw (2.8,2.4) node[anchor=north west] {$\pi$};
\draw [->] (1.2,0) -- (3.3,0);
\draw [->,line width=2pt,color=ffqqqq] (1.414,0.1) -- (1.414,0);
\draw [->,line width=2pt,color=ffqqqq] (pi,0.1) -- (pi,0);
\draw [color=ffqqqq](1.4,0.2) node[anchor=north west] {$\sqrt[]{2}$};
\draw [color=ffqqqq](3.1,0.2) node[anchor=north west] {$\pi$};
\end{tikzpicture}



Rationaalilukuja ja irrationaalilukuja kutsutaan yhdessä \emph{reaaliluvuiksi}. Reaalilukujen joukkoja merkitään $\mathbb{R}$.
Havainnollisesti reaalilukujen joukko sisältää kaikki lukusuoran luvut.

Rationaaliluvut yksinään eivät täytä koko lukusuoraa, mutta niitä on niin tiheässä, että jokaista irrationaalilukua voidaan approksimoida mielivaltaisen tarkasti jollain rationaaliluvulla. Esimerkiksi $\pi$ voidaan kirjoittaa halutusta tarkkuudesta riippuen seuraaviin muotoihin:
\[
3; \quad 3,1; \quad 3,14; \quad 3,142; \quad 3,1416; \quad 3,14159; \quad 3,141593; \quad \ldots 
\]



%\missingfigure{kuva tähän lukusuorasta, jolla on neliöjuuri 2 ja sen vieressä rationaalilukuja yhä lähempänä ja lähempänä -- ne approksimoivat neliöjuuri kakkosta}
\begin{center}
    \begin{lukusuora}{1.405}{1.425}{15}
    	\lukusuorapystyviiva{1.41}{$1,41$}
    	
    	\lukusuorapystyviiva{1.42}{$1,42$}
{\color{red}	\lukusuorapiste{1.414213562373}{$\sqrt{2}\ \ $}}
        \lukusuorapiste{1.413}{$\frac{1413}{1000}$}
        \lukusuorapiste{1.412}{$\frac{353}{250}$}
        \lukusuorapiste{1.415}{$\frac{283}{200}$}
        \lukusuorapiste{1.416}{$\frac{177}{125}$}
        \lukusuorapiste{1.418}{$\frac{709}{500}$}
      \end{lukusuora}
      
    \end{center}

Vaikka menisimme kuinka lähelle tahansa jotain irrationaalilukua lukusuoralla, välistä löytyisi vielä jokin rationaaliluku.


\laatikko{
Kaikki rationaalilukuja koskevat laskusäännöt pätevät myös reaaliluvuille.
}

Tämän perusteleminen on valitettavasti liian pitkällinen asia lukion ykköskurssille.

Siinä missä rationaalilukujen desimaaliesitykset ovat päättyviä tai jaksollisia, ovat irrationaalilukujen desimaaliesitykset päättymättömiä ja
jaksottomia. Monissa tapauksissa on hyvin vaikeaa osoittaa luku irrationaaliseksi. Tällöin vaaditaan keinoja, jotka eivät kuulu tämän kurssin laajuuteen.

Luvun
\[\sqrt{2} \approx 1,414213562373095048801688724209\ldots\]
desimaaliesityksessä on kyllä toistuvia kohtia, esimerkiksi numeropari $88$ esiintyy kahdesti. Siinä ei kuitenkaan ole jaksoa, jonka luvut toistuisivat yhä uudestaan
samassa järjestyksessä, kuten luvussa $3,80612312312312\ldots$.

Reaalilukujen ominaisuuksista kerrotaan lisää liitteessä \ref{aksioomat}.


Reaalilukujen myötä kaikki lukiokursseissa esiintyvät lukujoukot on nyt esitelty. Ne on lueteltu seuraavassa:
\begin{center}\begin{tabular}{l|c|l}
Joukko & Symboli & Mitä ne ovat\\
\hline
Luonnolliset luvut & $\mathbb{N}$ &
Luvut $0$, $1$, $2$, $3$, $\ldots$ \\
Kokonaisluvut & $\mathbb{Z}$ & Luvut $\ldots$ $-2$, $-1$, $0$, $1$, $2$ $\ldots$ \\ 
Rationaaliluvut & $\mathbb{Q}$ & Luvut, jotka voidaan esittää
murtolukuna \\
Reaaliluvut & $\mathbb{R}$ & Kaikki lukusuoran luvut \\
& & eli kaikki desimaaliluvut
\end{tabular} \end{center} 

\begin{tikzpicture}[line cap=round,line join=round,>=triangle 45,x=0.5cm,y=0.5cm]
\clip(-7.4,-8.8) rectangle (16.8,8.6);
\draw [rotate around={0.5:(2.2,0)}] (2.2,0) ellipse (1.1cm and 0.9cm);
\draw [rotate around={-0.8:(2.5,0)}] (2.5,0) ellipse (2cm and 1.6cm);
\draw [rotate around={-0.8:(2.5,0)}] (2.5,0) ellipse (2.9cm and 2.6cm);
\draw (2,1.5) node[anchor=north west] {$\mathbb{N}$};
\draw (4.0,2.7) node[anchor=north west] {$\mathbb{Z}$};
\draw (5.5,3.9) node[anchor=north west] {$\mathbb{Q}$};
%\draw (6.6,-4.6) node[anchor=north west] {{\scriptsize Irrationaaliluvut}};
\draw (9.5,6.4) node[anchor=north west] {$\mathbb{R}$};
\draw [rotate around={0.5:(4.4,0)}] (4.4,0) ellipse (5cm and 4.2cm);
%\draw [rotate around={18.2:(7.9,-5.4)}] (7.9,-5.4) ellipse (2.7cm and 0.6cm);
\draw (0.8,1.6) node[anchor=north west] {$1$};
\draw (1,-0.4) node[anchor=north west] {$5$};
\draw (2.4,-0.2) node[anchor=north west] {$101$};
\draw (4.8,0.7) node[anchor=north west] {$-5$};
\draw (1.2,-1.7) node[anchor=north west] {$0$};
\draw (1.2,3.1) node[anchor=north west] {$-14$};
%\draw (4.1,-1) node[anchor=north west] {$75$};
\draw (4.2,-2.8) node[anchor=north west] {$-\frac{1}{3}$};
\draw (6.6,1.4) node[anchor=north west] {$2\frac{1}{2}$};
\draw (-1.6,0.9) node[anchor=north west] {$-3$};
%\draw (0.4,-3.1) node[anchor=north west] {$-4$};
\draw (2.4,4.7) node[anchor=north west] {$2,6$};
\draw (-1.3,4.1) node[anchor=north west] {$\frac{5}{7}$};
\draw (-2.7,-1) node[anchor=north west] {$0,1$};
\draw (5.5,-5.9) node[anchor=north west] {$\pi$};
\draw (9.1,-3) node[anchor=north west] {$\sqrt[]{2}$};
%\draw (6,-4.7) node[anchor=north west] {$-\frac{\pi}{2}$};
%\draw (9.8,1.6) node[anchor=north west] {$\frac{5}{2}$};
\draw (10.5,3) node[anchor=north west] {$\frac{1}{\sqrt{2}}$};
%\draw (4.4,7.3) node[anchor=north west] {$3$};
\draw (-0.1,-5.3) node[anchor=north west] {$\sqrt{15}$};
%\draw (-4.9,1.5) node[anchor=north west] {$-5$};
\draw (11.8,-0.9) node[anchor=north west] {$-\frac{\pi}{2}$};
\draw (-0.6,6.8) node[anchor=north west] {$0,10110111011110\ldots$};
%\draw (-3.7,-3) node[anchor=north west] {$-3$};
\end{tikzpicture}

Lukualueita voidaan laajentaa lisää vielä tästäkin, esimerkiksi \emph{kompleksiluvuiksi}, jotka voidaan esittää tason pisteinä. Kompleksilukuja tarvitaan muun muassa insinöörialoilla yliopistoissa ja ammattikorkeakouluissa. Esimerkiksi vaihtosähköpiirien analyysissä, signaalinkäsittelyssä ja säätötekniikassa käytetään runsaasti kompleksilukuja. Kompleksiluvut ovat tärkeitä myös matematiikan tutkimuksessa itsessään.

\subsection*{Harjoitustehtäviä}

\begin{tehtava}
Ovatko seuraavat luvut rationaalilukuja vai irrationaalilukuja? Kunkin desimaalit
noudattavat yksinkertaista sääntöä.
\begin{enumerate}[a)]
\item $0,123456789101112131415 \ldots$
\item $2,415115115115115115115 \ldots$
\item $1,010010001000010000010 \ldots$
\end{enumerate}
\begin{vastaus}
a) irrationaaliluku \ b) rationaaliluku (jakso on 151) \ c) irrationaaliluku
\end{vastaus}
\end{tehtava}

\begin{tehtava}
Mikä on pienin lukua -3 suurempi luku \\
a) kokonaislukujen \ b) luonnollisten lukujen \ c) reaalilukujen joukossa?
\begin{vastaus}
a) -2 \ b) 0 \ c) Sellaista ei ole. Jos nimittäin $a > -3$, niin keskiarvo
$\frac{-3+a}{2}$ on vielä lähempänä lukua $-3$. 
\end{vastaus}
\end{tehtava}

\begin{tehtava}
$\boldsymbol{[\star]}$ Osoita, että \\
a) jokaisen kahden rationaaliluvun välissä on rationaaliluku \\
b) jokaisen kahden rationaaliluvun välillä on irrationaaliluku \\
c) jokaisen kahden irrationaaliluvun välissä on rationaaliluku \\
d) jokaisen kahden irrationaaliluvun välissä on irrationaaliluku. \\
e) Perustele edellisten kohtien avulla, että minkä tahansa kahden luvun
välissä on äärettömän monta rationaali- ja irrationaalilukua.
\begin{vastaus}
Vihjeet: a) keskiarvo \ b) $\sqrt{2}$ on irrationaaliluku. Käytä
painotettua keskiarvoa. \ c) pyöristäminen \ d) hyödynnä b-kohtaa
e) keskiarvot
\end{vastaus}
\end{tehtava}

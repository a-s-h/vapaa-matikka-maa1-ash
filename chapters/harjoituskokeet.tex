\chapter{Harjoituskokeita}

\section*{Harjoituskoe 1}

\begin{description}
	\item[] Ei laskinta.
	\item[1.] Kerro jokin \\
	(a) aina tosi yhtälö \\
	(b) joskus tosi yhtälö (ja milloin se on tosi) \\
	(c) aina epätosi yhtälö.
	\item[2.] Laske \\
	(a) $\sqrt{144}$ \qquad
	(b) $\sqrt{196}$ \qquad
	(c) $\sqrt[3]{64}$ \qquad
	(d) $\sqrt[3]{216}$.
	\item[3.] Muuta murtoluvuksi \\
	(a) $0,45$ \qquad
	(b) $0,33\ldots$ \qquad
	(c) $0,2857142\ldots$ \qquad
	(d) $0,388\ldots$.
	\item[4.] Laske auki \\
	(a) $(a+b)^2$ \qquad
	(b) $(a-b)^2$ \qquad
	(c) $(a+b)(a-b)$.
	\item[5.] Mitkä seuraavista luvuista ovat alkulukuja? \\
	(a) $11$ \qquad
	(b) $4$ \qquad
	(c) $29$ \qquad
	(d) $39$.
	\item[6.] Ratkaise \\
	(a) $x^2 = 1$ \qquad
	(b) $x^4 = x^2$.
	\item[7.] Muuta seuraavat kymmenjärjestelmän luvut heksadesimaaliluvuiksi. \\
	(a) $175$ \qquad
	(b) $384$.
	\item[8.] Funktio $f$ määritellään kaavalla $f(x) = x^2 + 2x + 3$. Ilmaise $f(f(x))$ muodossa, jossa ei ole termiä $f(x)$.
\end{description}

\section*{Harjoituskoe 2}

\begin{description}
	\item[] Ei laskinta.
	\item[1.] Kerro jokin \\
	(a) kokonaisluku, joka ei ole luonnollinen luku \\
	(b) rationaaliluku, joka ei ole kokonaisluku \\
	(c) reaaliluku, joka ei ole rationaaliluku.
	\item[2.] Ratkaise \\
	(a) $11x=77$ \qquad
	(b) $8x+174=50$ \\
	(c) $7x+7=5x$ \qquad
	(d) $6x+7=8x+1$.
	\item[3.] Muuta desimaaliluvuksi \\
	(a) $\frac{3}{10}$ \qquad
	(b) $\frac{3}{16}$ \qquad
	(c) $\frac{5}{9}$ \qquad
	(d) $\frac{5}{11}$.
	\item[4.] Tuoreessa ananaksessa veden osuus on 80\% ananaksen massasta ja A-, B- ja C-vitamiinien yhteenlaskettu osuus 0,05\% massasta. Ananas kuivatetaan niin, että veden osuus laskee 8 prosenttiin ananaksen massasta. Kuinka suuri on A-, B- ja C-vitamiinien osuus kuivatun ananaksen massasta? (Luvut eivät ole faktuaalisia.)
	\item[5.] Muuta seuraavat kymmenjärjestelmän luvut binääriluvuiksi. \\
	(a) $100$ \qquad
	(b) $128$.
	\item[6.] Sievennä \\
	(a) $\frac{a^2 b^2}{a}$, $a \neq 0$ \qquad
	(b) $3(a^2+1)-2(a^2-1)$ \\
	(c) $ab(a+2a)$ \qquad
	(d) $(a^3 b^2 c)^2$.
	\item[7.] Olkoon $f(t) = 35 \cdot 2^t$ bakteerien lukumäärä soluviljelmässä ajanhetkellä $t$ (sekuntia). Monenko sekunnin kuluttua bakteereita on yli 1000? Yhden sekunnin tarkkuus ylöspäin pyöristettynä riittää.
	\item[8.] Määritellään kahdelle järjestetylle lukunelikolle $(a, b, c, d)$ laskutoimitus $\odot$ seuraavasti: $(a_1, b_1, c_1, d_1) \odot (a_2, b_2, c_2, d_2) = (a_1 a_2 + b_1 c_2, a_1 b_2 + b_1 d_2, c_1 a_2 + d_1 c_2, c_1 b_2 + d_1 d_2)$. Laske $(1, 1, 1, 0) \odot (1, 1, 1, 0)$.

\end{description}
